\documentclass[a4paper]{article}
\usepackage[english]{babel}
\usepackage{setspace}
\usepackage{mathtools}
\usepackage{listings}
\usepackage[utf8]{inputenc}
\usepackage{eurosym}
\usepackage{fancyhdr}
\usepackage{tikz}
\usetikzlibrary{calc}
\usetikzlibrary{positioning}
\usetikzlibrary{arrows.meta}
\usetikzlibrary{automata}
\usetikzlibrary{fit}
\usepackage{tikzscale}
\usepackage{wrapfig}
\setcounter{secnumdepth}{-1}
\pagestyle{fancy}
\fancyhf{}
\setlength{\headheight}{24.0pt}
\lhead{Multi-Agent Systems,  Winter Semester 2018/2019, Exercise 11\\
       Submitted by: Amadeus Hovekamp, Hans Nübel, Jonathan Pieper}
\cfoot{\thepage}

\usepackage[
        colorlinks = true,    % Disable drawing boxes around links.
        linkcolor = black,    % Sets the color of links to black.
    ]{hyperref}
\definecolor{darkgreen}{rgb}{0.0, 0.5, 0.0}
\definecolor{darkred}{rgb}{0.5, 0.0, 0.0}
\newcommand{\refequation}[1]{\hyperref[#1]{(\ref{#1})}}

\begin{document}
\title{Exercise Sheet 11: Solution}
\author{}
\date{\today}

\section{Exercise 11.1}
\subsection{a)}
\begin{tabular}{r|l}
Step    & L\\\hline
0       & $\emptyset$\\
1       & $\{(a,\textbf{in})\}$\\
2       & $\{(a,\textbf{in}), (b,\textbf{out})\}$\\
3       & $\{(a,\textbf{in}), (b,\textbf{out})\}$\\
\end{tabular}

The grounded labeling is $L_1 := \{(a,\textbf{in}), (b,\textbf{out}), (c,\textbf{undec}), (d,\textbf{undec}), (e,\textbf{undec})\}$

\subsection{b)}
The grounded labeling $L_1$ is (by definition) complete. Further complete labelings are:\\
$L_2 := \{(a,\textbf{in}), (b,\textbf{out}), (c,\textbf{out}), (d,\textbf{in}), (e,\textbf{out})\}$\\
$L_3 := \{(a,\textbf{in}), (b,\textbf{out}), (c,\textbf{in}), (d,\textbf{out}), (e,\textbf{in})\}$\\
$L_2$ and $L_3$ are both preferred and stable.

\subsection{c)}
\begin{itemize}
\item \emph{e} is in the \textbf{in} set of some preferred labeling\\
M: in(e)\\
S: out(d)\\
M: in(c)\\
S: out(b)\\
M: in(a)
\item \emph{d} is in the \textbf{in} set of some preferred labeling\\
M: in(d)\\
S: out(c)\\
M: in(d)\\
\item \emph{c} is in the \textbf{in} set of some preferred labeling\\
M: in(c)\\
S: out(b)\\
M: in(a)\\
S: out(d)\\
M: in(c)\\
\item \emph{b} is not in the \textbf{in} set of some preferred labeling\\
M: in(b)\\
S: out(a)\\
\item \emph{a} is in the \textbf{in} set of some preferred labeling\\
M: in(a)\\
\end{itemize}



\end{document}
