\documentclass[a4paper]{article}
\usepackage[english]{babel}
\usepackage{setspace}
\usepackage{mathtools}
\usepackage{listings}
\usepackage{ulem}
\usepackage[utf8]{inputenc}
\usepackage{eurosym}
\usepackage{amssymb}
\usepackage{fancyhdr}
\usepackage{tikz}
\usepackage{tikzsymbols}
\usetikzlibrary{calc}
\usetikzlibrary{positioning}
\usetikzlibrary{arrows.meta}
\usetikzlibrary{fit}
\usepackage{tikzscale}
\usepackage{wrapfig}
\setcounter{secnumdepth}{-1}
\pagestyle{fancy}
\fancyhf{}
\setlength{\headheight}{24.0pt}
\lhead{Multi-Agent Systems,  Winter Semester 2018/2019, Exercise 5\\
       Submitted by: Amadeus Hovekamp, Hans Nübel, Jonathan Pieper}
\cfoot{\thepage}

\usepackage[
        colorlinks = true,    % Disable drawing boxes around links.
        linkcolor = black,    % Sets the color of links to black.
    ]{hyperref}
\newcommand{\refequation}[1]{\hyperref[#1]{(\ref{#1})}}

\begin{document}
\title{Exercise Sheet 5: Solution}
\author{}
\date{\today}

\section{Exercise 5.1}
\subsection{a)}
$
K(\phi \land \lnot K \phi)
\models_{N, K} K \phi \land K \lnot K \phi
\models_T K \phi \land \lnot K \phi
\models \bot
$\\
An agent cannot know know $\phi$ while not knowing $\phi$.\\
Rather, $[!\phi \land \lnot K\phi]K\phi \land K\phi$ holds.

\subsection{b)}
$
K_j K_i \phi
\models K_j K_i \phi \land (K_i \phi \rightarrow_T \phi)
\models_N K_j K_i \phi \land K_j(K_i \phi \rightarrow_T \phi)
\models_K K_j \phi
$\\
An agent knows all logical implications of his knowledge, including the use of axioms. 

\subsection{c)}
$
O(\phi \land \psi)
\models O(\phi \land \psi) \land (\phi \land \psi \rightarrow \psi)
\models_N O(\phi \land \psi) \land O(\phi \land \psi \rightarrow \psi)
\models_K O(\psi)
$\\
An agent is obliged to all generalizations of his obligations.

\subsection{d)}
$
P\phi \land \lnot P\psi
\models P\phi
\models_{N, K} P(\phi \lor \psi)
$\\
If only $\phi$ is permitted while $\psi$ is not, $\phi \lor \psi$ is permitted (weakening rule). This is a counterexample to the given statement.
\subsection{e)}
$
\models_T O(\phi) \rightarrow \phi
\models_N O(O(\phi) \rightarrow \phi)
$

\section{Exercise 5.2}
$p$ is true iff the passenger is alive. $q$ is true iff the pedestrian is alive.\\
\subsection{a)}
$
\Diamond(K p \land K \lnot q) \land \Diamond(\lnot K p \land \lnot K \lnot p \land K q)
$
\subsection{b)}
$
O(\lnot K p \land \lnot K \lnot p \land K q)
$
\subsection{c)}
$
O(\lnot K \lnot p \land \lnot K \lnot q)
$
\subsection{d)}
$
O(\lnot K p \land \lnot K \lnot p \land K q)
\models_{N, K} O(\lnot K \lnot p \land K q)
\models_{D, N, K} O(\lnot K \lnot p \land \lnot K \lnot q)
$
\end{document}
