\documentclass[a4paper]{article}
\usepackage[english]{babel}
\usepackage{setspace}
\usepackage{mathtools}
\usepackage{listings}
\usepackage{ulem}
\usepackage[utf8]{inputenc}
\usepackage{eurosym}
\usepackage{amssymb}
\usepackage{fancyhdr}
\usepackage{tikz}
\usepackage{tikzsymbols}
\usetikzlibrary{calc}
\usetikzlibrary{positioning}
\usetikzlibrary{arrows.meta}
\usetikzlibrary{fit}
\usepackage{tikzscale}
\usepackage{wrapfig}
\setcounter{secnumdepth}{-1}
\pagestyle{fancy}
\fancyhf{}
\setlength{\headheight}{24.0pt}
\lhead{Multi-Agent Systems,  Winter Semester 2018/2019, Exercise 7\\
       Submitted by: Amadeus Hovekamp, Hans Nübel, Jonathan Pieper}
\cfoot{\thepage}

\usepackage[
        colorlinks = true,    % Disable drawing boxes around links.
        linkcolor = black,    % Sets the color of links to black.
    ]{hyperref}
\newcommand{\refequation}[1]{\hyperref[#1]{(\ref{#1})}}

\begin{document}
\title{Exercise Sheet 7: Solution}
\author{}
\date{\today}

\section{Exercise 7.1}
\subsection{a)}
Since there is enough space, Car 3 performs an acceleration, that is possibly taken back due to randomization. For $p=0$, the value for $v_3$ in the next iteration is always $3$, for $p=1$, it is always $2$ and for $p=0.25$ it is either of these two.

\subsection{b)}
\begin{tabular}{|p{.4cm}|p{.4cm}|p{.4cm}|p{.4cm}|p{.4cm}|p{.4cm}|p{.4cm}|p{.4cm}|p{.4cm}|p{.4cm}|p{.4cm}|p{.4cm}|p{.4cm}|p{.4cm}|p{.4cm}|p{.4cm}|}
\hline
&&\ X \tiny$v_1$=$2$&&&&&&&\ X \tiny $v_2$=$5$&&&&\ X \tiny $v_3$=$3$&\ X \tiny $v_4$=$0$&\\
\hline
\end{tabular}\\
\\
\begin{tabular}{|p{.4cm}|p{.4cm}|p{.4cm}|p{.4cm}|p{.4cm}|p{.4cm}|p{.4cm}|p{.4cm}|p{.4cm}|p{.4cm}|p{.4cm}|p{.4cm}|p{.4cm}|p{.4cm}|p{.4cm}|p{.4cm}|}
\hline
&&&&&\ X \tiny$v_1$=$3$&&&&&&&\ X \tiny $v_2$=$3$&\ X \tiny $v_3$=$0$&&\ X \tiny $v_4$=$1$\\
\hline
\end{tabular} 


\end{document}
